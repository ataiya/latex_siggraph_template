% !TEX root = ../htrack.tex

% Things to include in the intro are (not necessarily in this order):
% 
% Problem Statement: What's the problem you want to solve?
% 
% Motivation: Why is this an interesting problem? Who cares about it? Why now? Why is it appropriate for the conference audience?
% 
% Research Gap, Novelty: Why is new research required? Why can the problem not be solved with existing methods? How does the proposed solution differ from and/or improve upon existing work?
% 
% Technical Contribution: What's the key technical idea to solve the problem? Why is it beautiful?
% 
% Applications / Future Work: What will your solution enable? How does it project into the future? How will it inspire future work?

\section{Introduction}
Lorem ipsum dolor sit amet, consectetur adipisicing elit, sed do eiusmod tempor incididunt ut labore et dolore magna aliqua. Ut enim ad minim veniam, quis nostrud exercitation ullamco laboris nisi ut aliquip ex ea commodo consequat. Duis aute irure dolor in reprehenderit in voluptate velit esse cillum dolore eu fugiat nulla pariatur. Excepteur sint occaecat cupidatat non proident, sunt in culpa qui officia deserunt mollit anim id est laborum.

\AT{This is what I've been using in many groups to do inline annotations}. See the tweaks.tex file to know what is your id (GM, AT, NP, PC, EP), for example, for giorgio: \GM{ciao ciao!!}.

\begin{figure}[t]
\centering
\begin{overpic} 
% [width=\linewidth]
[width=\linewidth,grid,tics=10]
{\currfiledir/item.pdf}
\put(10,10){\todo{\currfiledir}}
\end{overpic}
\caption{\todo{\currfiledir}}
\label{fig:onecol}
\end{figure}
% \filename@parse{path/to/file.c}